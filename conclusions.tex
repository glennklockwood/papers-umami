\section{Conclusions and Future Work} \label{sec:conclusions}

Attempting to understand I/O behavior on modern-day HPC systems is a daunting task, due in large part to the increasingly complex and hierarchical nature of the underlying I/O subsystem.
In this work we have presented TOKIO, a framework providing holistic instrumentation and analysis of HPC I/O subsystem behavior by collecting and integrating data from numerous components throughout the system.
We have leveraged the TOKIO framework to investigate I/O subsystem behavior on two leadership-class HPC systems, with each system subjected to daily I/O performance regression benchmarking over a 1-month period.
We used the data generated by this experiment to first characterize the I/O \emph{climate} on each system, and then to detect and investigate particular instances of abnormal I/O \emph{weather}.

We were able to utilize the TOKIO framework to make a number of notable insights into the I/O climate of the two HPC systems analyzed through the course of this study:

\begin{itemize}
\item I/O performance variability is uniquely affected by both intrinsic application behavior and extrinsic storage system factors
\item Contention with other I/O workloads for storage system bandwidth is not the only factor that affects I/O variability; we observed numerous instances of jobs that had uninhibited access to storage system resources that still exhibited poor I/O performance
\item There is no single monitoring metric that most correlates with I/O performance universally across HPC platforms; the metrics that correlate most with I/O performance depend not only on system architecture, but also on I/O subsystem configuration parameters, workload characteristics, health, etc.
\end{itemize}

We found the following about individual jobs:

\begin{itemize}
\item individual job performance problems can be more rapidly diagnosed if the performance metrics are presented in context with historical and holistic I/O characterization (kind of wrapping up the overarching point of the 3 job-oriented case studies)
\item integrating monitoring metrics into a coherent framework with normalized data not only simplifies analysis, but also makes it possible to readily share data and methods across computing facilities, thereby enabling previously infeasible comparative studies and shared analysis methods (too wordy, but this is kind of the meta point about the value of being able to tie this  stuff together into the generalized umami tool, we've demonstrated it by applying it on two very different platforms)
\end{itemize}

In the future we will...

\todo{It would be great if there were some tangible artifacts from this work.
Possible examples:}
\begin{itemize}
\item open repo for benchmark configs and cron jobs so others can replicate
performance regression testing
\item anonymized data collected in study
\item new data collection tools (LMT monitoring, Lustre failover monitoring,
mmpmon monitoring, etc.)
\end{itemize}

%\todo{finish writing prose...} We have demonstrated the TOKIO framework as a powerful framework for integrating monitoring data from components throughout the I/O stack.  In combination with the TOKIO Automated Benchmark Collection, we have confirmed that contention for both bandwidth, IOPs, metadata operations, and system-specific metrics all play a role in I/O variability.  Furthermore, we have shown that 



% Overall, instances of abnormally poor I/O performance on Mira were most often attributable to equally abnormal coverage factors (indicating bandwidth contention) or high metadata rates (opens/closes and readdirs).
% Of all \todo{UPDATE}92 tests run on this file system, only one case of performance falling in the first quartile was not accompanied by one of these server-side contention conditions.
% The most common sources of performance loss on Edison were more difficult to neatly categorize because server-side IOPS were not measured explicitly.
% In those cases where performance was poor despite a high coverage factor, many secondary indicators of high IOPS load (such as high data server CPU load and high metadata rates) were observed.

% Qualitatively, the highly variable I/O workload and "I/O weather" conditions across the Edison file systems obfuscate a straightforward analysis of all of the sources of performance variation on its file systems.
% Over the course of this study, we found that the TOKIO framework often exposed variation resulting from the confluence of several factors that only correlate weakly with performance when examined independently.
% Thus, TOKIO requires some amount of expert knowledge to confidently determine the root cause of I/O problems on file systems that are subject to large amounts of incoherent or otherwise noisy I/O.
% That said, presenting the historic correlations between performance and each variable through UMAMI does dramatically reduce the exploration space in this effort.
%
% Capturing an historic record of a file system's climate for a particular workload is not always straightforward, and at present, users will have to self-identify jobs and Darshan logs that should be used to build this historic record.
% Alternatively, UMAMI can be seeded with data from baseline performance tests, as was done in this study, provided that the user's I/O motif is sufficiently similar to one of the baseline tests' motifs.
% This can be automated to a large degree though; the simplest approach is to scan Darshan logs for similar I/O transaction size distributions as seeds.
% Recently, Liu et al have also demonstrated combining resource manager logs and server-side I/O logs to identify a series of I/O bursts with specific jobs using a density-based clustering method~\cite{Liu2016}.
% We believe that adapting this clustering-based classification approach with TOKIO-UMAMI should be a straightforward approach to building an I/O climate record for any arbitrary job.
% 
% Finally, although we applied TOKIO and UMAMI to build I/O climate histories specific to applications based on Darshan logs, it is equally possible to group historic data around similar periods of server-side traffic.
% For example, TOKIO can be used to identify collections of applications that, when run together, result in an uncommonly adversarial I/O load or, conversely, surprisingly good performance.
% From this, it is easy to envision an opportunity to inform job scheduling to avoid I/O contention between two frequently conflicting applications.
