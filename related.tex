\section{Related work} \label{sec:related}

\assign{Phil} \todo{include SIOX} Let's cite Xiaosong Ma's work in server-side
monitoring \cite{Liu2016}; her work uses server-side logs to make scheduling
recommendations based on a knowledgebase of applications and their historic
I/O requirements.  This work can improve the predictive ability of such systems
by replacing the black-box weighting parameter, $w_{i}$, with a function that
accurately captures the effects of different external interference sources for
an application.

Matthieu showed how to minimize I/O jitter~\cite{Dorier2012} (but I haven't read
this paper yet...).  More recently, Yildiz et al~\cite{Yildiz2016} performed
a systematic exploration of points of contention on an idealized system and
exposed the sensitivity of 10 GbE to various forms of interference.

Lofstead et al wrote a paper on managing I/O variability in production~
\cite{Lofstead2010} which characterized \emph{internal interference} and
\emph{external interference} using purely client-side metrics.  They defined an
\emph{imbalance factor}, the ratio of the slowest to fastest write times, as a
means to infer the external interference, but we can precisely quantify it here
with server-side metrics.

Andrew Uselton's work on understanding I/O performance in terms of ensembles of
bursts might be relevant\cite{Uselton2010}, but his method requires heavyweight
I/O tracing to determine the statistical distribution of I/O bursts during an
application execution and, as such, as better suited to characterizing the
behavior of a specific file system as a one-time activity.

A long time ago, David Skinner and Bill Kramer wrote a paper that characterized
sources of performance variation~\cite{Skinner2005}, but it didn't really talk
about I/O in a very meaningful way.

More recently Xie et al also characterized I/O problems at Oak
Ridge\cite{Xie2012} using IOR and quantified the effects of stripe counts,
straggling writers, and shared-file I/O to target areas where middleware can
optimize I/O for applications.

Is any of the BeeGFS stuff (dynamically provisioning file systems) relevant to
interference isolation?

\section{Conclusions} \label{sec:conclusions}

\todo{It would be great if there were some tangible artifacts from this work.
Possible examples:}
\begin{itemize}
\item open repo for benchmark configs and cron jobs so others can replicate
performance regression testing
\item anonymized data collected in study
\item new data collection tools (LMT monitoring, Lustre failover monitoring,
mmpmon monitoring, etc.)
\end{itemize}
