\section{Related Work} \label{sec:related}

Several studies have explored how to combine and analyze multiple sources of I/O monitoring information.
Kunkel et al. developed SIOX~\cite{Kunkel:2014:SAC:2769884.2769901} to aggregate information from multiple components of the I/O stack, 
but its reliance on instrumented versions of application libraries to collect metrics makes production deployment challenging.
Liu et al. developed AID~\cite{Liu2016} to perform detailed analysis to server-side I/O logs in order to deduce application-level I/O patterns and make scheduling recommendations.  The clustering approach implemented by AID may be able to automatically identify I/O motifs suitable for generating UMAMI diagrams.

Other studies have explored how to quantify and combat various types of I/O performance variation.
Lofstead et al. observed that variability can be caused by both external interference and internal interference within an application~\cite{Lofstead2010}, and they proposed an adaptive strategy that coordinates I/O activity within an application.
Similarly, Dorier et al. proposed a middleware mechanism for coordinating I/O activity across applications to manage external interference~\cite{dorier2014calciom}.
Yildiz et al.'s systematic study of I/O interference in a controlled testbed environment found that poor flow control in the I/O path~\cite{Yildiz2016} also contributes to I/O variance, indicating that network-related metrics would be a valuable addition to UMAMI diagrams.
Carns et al. reported I/O performance variability for seven frequently repeated production jobs during a two-month study~\cite{carns2011understanding} and, similar to our findings, suggested that some access patterns are more susceptible to variability than are others.
