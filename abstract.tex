\begin{abstract}

I/O efficiency is essential to productivity in scientific computing, especially as many scientific domains become more data-intensive.
A variety of characterization tools have been used to elucidate and optimize specific aspects of parallel I/O.
However, analyzing individual component of complex I/O subsystems in isolation fails to provide insight into the most important questions:
how do the I/O components interact,
what are reasonable expectations for application performance, and what are the underlying causes and effects of I/O performance problems?
These questions suggest the the need for a unified, comprehensive, multi-level, system-wide I/O characterization tool, but such an effort would fail to capitalize on existing proven component-level characterization tools, and it would not be portable across facilities with diverse computing and storage vendors.

In this work, we instead evaluate the feasibility of a practical alternative approach: on-demand, modular synthesis of \emph{existing} I/O characterization data sources into a normalized, coherent, holistic view of I/O behavior.
We explore this approach on two distinct large-scale computing platforms with three benchmark case studies.
The case studies highlight the importance of analyzing application I/O performance in context with both contemporaneous and historical component metrics.
The case studies also reveal the challenges of portable data integration and the potential impact of applying these techniques to production workloads.

% I/O efficiency is essential to productivity in scientific computing,
% especially as many scientific domains become more data-intensive.
% A variety of characterization tools have been used to elucidate and optimize specific aspects of parallel I/O.
% However, analyzing individual component of the I/O subsystem in isolation fails to provide insight into the most important questions:
% how do the I/O components interact,
% what are reasonable expectations for application performance,
% and what are the underlying causes and effects of I/O performance problems?

% In this work we present the results of holistic I/O characterization that integrates I/O instrumentation data from multiple sources to provide novel insight into performance.
% We incorporate data from file system instrumentation, application instrumentation, and health monitoring on two distinct, large-scale computing platforms and 
% use periodic regression benchmarking to convert disparate sources of data into knowledge about system-level performance variation.
% We then present three case studies that highlight how this holistic approach can be used to understand abnormal I/O performance in increasingly complex storage systems.

\end{abstract}
