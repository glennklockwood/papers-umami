\begin{abstract}
%%% I had to submit a 150-word abstract to SC, so below is the shortened
%%% version I submitted
%
% As scientific computing becomes more data-intensive and storage hierarchies
% deepen, I/O performance is becoming increasingly critical to productivity in
% HPC.  Instrumentation and analysis tools have long been used to understand
% parallel I/O performance, but analyzing the individual components of I/O
% subsystems in isolation fails to answer how these components interact and
% result in poor I/O performance.
%
% To this end, we have developed a framework for holistic I/O characterization
% that integrates instrumentation from file system servers, applications, file
% system health monitors, and other system resources.  Along with formalized
% periodic regression benchmarking, we then demonstrate this framework's
% portability and unobtrusiveness by deploying it in production on two
% leadership-class computing platforms.  Using data collected during a month
% long study, we provide unique insights into applications' I/O performance
% that are enabled by this approach.  We then extend these findings to provide
% a broader understanding of how application performance varies across
% different file systems and workloads.

I/O efficiency is essential to productivity in scientific computing,
especially as more scientific domains become more data-intensive and
large-scale computing platforms incorporate more complex storage
hierarchies.
A variety of instrumentation and analysis tools have been
utilized to great effect to help understand and optimize specific aspects of
HPC I/O, such as application access patterns, storage device traffic, and
distributed file system configurations.
However, analyzing individual services in the
I/O ecosystem in isolation fails to provide insight into the most important
questions: how do the I/O components interact, what are reasonable expectations for application performance, and what are the underlying causes and effects of I/O performance problems?

In this work we propose a framework for holistic I/O characterization that combines I/O instrumentation data from multiple sources to obtain insights that were previously unobtainable.
We describe a methodology that incorporates data from file system instrumentation, application instrumentation, health monitoring, and other sources, and then demonstrate its applicability, portability, and unobtrusiveness of this approach by deploying it in production on two distinct leadership-class computing platforms.
Using formalized periodic regression benchmarking, we apply our holistic I/O characterization methodology to convert disparate sources of data into knowledge about system-level performance variation.
We then use several case studies to highlight how this framework can be used to improve our understanding of scientific application performance.

\end{abstract}
