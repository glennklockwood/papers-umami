\begin{abstract}

I/O efficiency is essential to productivity in scientific computing, especially as many scientific domains become more data-intensive.
Many characterization tools have been used to elucidate specific aspects of parallel I/O performance, but analyzing individual components of complex I/O subsystems in isolation fails to provide insight into critical questions:
how do the I/O components interact, what are reasonable expectations for application performance, and what are the underlying causes and effects of I/O performance problems?
To address these questions while capitalizing on existing component-level characterization tools,
we propose an approach using on-demand, modular synthesis of \emph{existing} I/O characterization data sources into a unified monitoring and metrics interface (UMAMI) that provides a normalized, coherent, holistic view of I/O behavior. 

In this work, we evaluate the feasibility of this integrative approach by applying it to a month-long benchmarking study conducted on two distinct large-scale computing platforms.
We present three case studies which highlight the importance of analyzing
application I/O performance in context with both contemporaneous and
historical component metrics, and we provide new insights into the factors
affecting I/O performance and how to analyze them.

\end{abstract}
