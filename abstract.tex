\begin{abstract}

I/O efficiency is essential to productivity in scientific computing, especially as many scientific domains become more data-intensive.
Many characterization tools have been used to elucidate specific aspects of parallel I/O performance, but analyzing components of complex I/O subsystems in isolation fails to provide insight into critical questions:
how do the I/O components interact, what are reasonable expectations for application performance, and what are the underlying causes of I/O performance problems?
To address these questions while capitalizing on existing component-level characterization tools,
we propose an approach that combines on-demand, modular synthesis of I/O
characterization data into a unified monitoring and metrics interface
(UMAMI) to provide a normalized, holistic view of I/O behavior. 

We evaluate the feasibility of this approach by applying it to a month-long benchmarking study on two distinct large-scale computing platforms.
We present three case studies that highlight the importance of analyzing application I/O performance in context with both contemporaneous and historical component metrics, and we provide new insights into the factors affecting I/O performance.
%\todo{Something about turning this into a production tool.}
By demonstrating the generality of our approach, we lay the groundwork for a production-grade framework for holistic I/O analysis.

\end{abstract}
