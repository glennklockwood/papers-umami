\documentclass[conference,10pt,compsocconf]{IEEEtran}

% *** CITATION PACKAGES ***
%
\usepackage{cite}
% cite.sty was written by Donald Arseneau
% V1.6 and later of IEEEtran pre-defines the format of the cite.sty package
% \cite{} output to follow that of IEEE. Loading the cite package will
% result in citation numbers being automatically sorted and properly
% "compressed/ranged". e.g., [1], [9], [2], [7], [5], [6] without using
% cite.sty will become [1], [2], [5]--[7], [9] using cite.sty. cite.sty's
% \cite will automatically add leading space, if needed. Use cite.sty's
% noadjust option (cite.sty V3.8 and later) if you want to turn this off.
% cite.sty is already installed on most LaTeX systems. Be sure and use
% version 4.0 (2003-05-27) and later if using hyperref.sty. cite.sty does
% not currently provide for hyperlinked citations.
% The latest version can be obtained at:
% http://www.ctan.org/tex-archive/macros/latex/contrib/cite/
% The documentation is contained in the cite.sty file itself.
%
\usepackage{setspace}
\usepackage{wrapfig}
\usepackage[usenames, dvipsnames]{color}
\usepackage{balance}

%\usepackage[pdftex]{graphicx}
\usepackage{graphicx}
% declare the path(s) where your graphic files are
\graphicspath{{./}{./figs}}
% and their extensions so you won't have to specify these with
% every instance of \includegraphics
\DeclareGraphicsExtensions{.pdf,.jpeg,.png}

% *** SUBFIGURE PACKAGES ***
\usepackage[tight,footnotesize]{subfigure}
% \usepackage{subfigure}
% subfigure.sty was written by Steven Douglas Cochran. This package makes it
% easy to put subfigures in your figures. e.g., "Figure 1a and 1b". For IEEE
% work, it is a good idea to load it with the tight package option to reduce
% the amount of white space around the subfigures. subfigure.sty is already
% installed on most LaTeX systems. The latest version and documentation can
% be obtained at:
% http://www.ctan.org/tex-archive/obsolete/macros/latex/contrib/subfigure/
% subfigure.sty has been superceeded by subfig.sty.

%\usepackage[caption=false]{caption}
%\usepackage[font=footnotesize]{subfig}
% subfig.sty, also written by Steven Douglas Cochran, is the modern
% replacement for subfigure.sty. However, subfig.sty requires and
% automatically loads Axel Sommerfeldt's caption.sty which will override
% IEEEtran.cls handling of captions and this will result in nonIEEE style
% figure/table captions. To prevent this problem, be sure and preload
% caption.sty with its "caption=false" package option. This is will preserve
% IEEEtran.cls handing of captions. Version 1.3 (2005/06/28) and later
% (recommended due to many improvements over 1.2) of subfig.sty supports
% the caption=false option directly:
%\usepackage[caption=false,font=footnotesize]{subfig}
%
% The latest version and documentation can be obtained at:
% http://www.ctan.org/tex-archive/macros/latex/contrib/subfig/
% The latest version and documentation of caption.sty can be obtained at:
% http://www.ctan.org/tex-archive/macros/latex/contrib/caption/

% *** PDF, URL AND HYPERLINK PACKAGES ***
%
\usepackage{url}
% url.sty was written by Donald Arseneau. It provides better support for
% handling and breaking URLs. url.sty is already installed on most LaTeX
% systems. The latest version can be obtained at:
% http://www.ctan.org/tex-archive/macros/latex/contrib/misc/
% Read the url.sty source comments for usage information. Basically,
% \url{my_url_here}.

% *** Do not adjust lengths that control margins, column widths, etc. ***
% *** Do not use packages that alter fonts (such as pslatex).         ***
% There should be no need to do such things with IEEEtran.cls V1.6 and later.
% (Unless specifically asked to do so by the journal or conference you plan
% to submit to, of course. )

\newcommand{\assign}[1]{\textcolor{red}{(#1)}}
\newcommand{\todo}[1]{\textcolor{Orange}{TODO: #1}}

\begin{document}
\title{HPC I/O Ecosystem Instrumentation: Insights from correlating data}

\maketitle

\begin{abstract}

I/O efficiency is essential to productivity in scientific computing,
especially as most scientific domains become more data-intensive and
new large-scale computing platforms incorporate more complex storage
hierarchies.  A variety of instrumentation and analysis tools have been
utilized to great effect to help understand and optimize specific aspects of
HPC I/O, such as application access patterns, storage device traffic, and
distributed file system configurations.  Analyzing individual aspects of
the I/O ecosystem in isolation provides limited insight into the system as a
whole, however: how do the I/O components interact, what \emph{combinations}
of optimizations across the stack are most effective, and what are the
underlying causes and effects of I/O performance problems.

In this work we explore the potential for holistic I/O characterization
by combining I/O instrumentation data from multiple sources to obtain
insights that were previously unobtainable. We describe a methodology that
incorporates file system instrumenatation, application instrumentation,
fault monitoring, and formalized periodic regression benchmarking as
the foundation of portable I/O instrumentation, and then demonstrate
its applicability, portability, and inobtrusiveness by deploying that
methodology in production on two distinct leadership-class computing
platforms. Based on our \todo{some time period} study we observe
\todo{some outcome}.

\end{abstract}

\section{Introduction \assign{All}}

\emph{From Rob}: I think a component of the story is that when we approach these problems, there are a few challenges:
\begin{enumerate}
\item increasing number of interoperating components (in this case, additional
BB and DVS and so forth)
\item different components have different "views" on I/O, different levels of
monitoring, some of which aren't practical in production
\item no current framework for integration, lots of expert knowledge to
construct the story of what happened and how to fix.
\end{enumerate}
Cite something to get bibtex working for now~\cite{carns200924}.

\section{Instrumentation methods}

Brief description of tools that we are using.
\todo{overview diagram here?}

\subsection{Darshan \assign{Shane}}

\subsection{LMT \assign{Glenn}}

\subsection{mmpmon \assign{Phil}}

\subsection{fault monitoring \assign{Glenn}}

\todo{Cron job to standardize data about what servers are up/down/failedover over
time.}
\todo{Can we put capacity monitoring in here too?  Very similar.}

\section{Platforms and workloads}

We employ our methodology on two platforms:

\begin{itemize}
\item \textbf{Cori} \assign{Glenn}
\item \textbf{Mira} \assign{Phil}
\end{itemize}

We selected 4 represenative application workloads for periodic regression
benchmarking:

\begin{itemize}
\item \textbf{HACC} \assign{Suren}
\item \textbf{VPIC} \assign{Suren}
\item \textbf{Boxlib} \assign{Glenn} \todo{emphasize why this is interesting
because of anticipated challenges in AMR codes}
\item \textbf{IOR} \assign{Shane}
\end{itemize}

\todo{Describe how we are scheduling these jobs to run.} \assign{Glenn}

\section{Case studies}

\section{Conclusions}


\bibliographystyle{IEEEtran}
\bibliography{REFERENCES}


\end{document}
