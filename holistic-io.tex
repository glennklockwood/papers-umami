\documentclass[sigconf,10pt]{acmart}
%\documentclass[conference,10pt,compsocconf]{IEEEtran}

\let\labelindent\relax

% *** CITATION PACKAGES ***
%
%\usepackage{cite}
% cite.sty was written by Donald Arseneau
% V1.6 and later of IEEEtran pre-defines the format of the cite.sty package
% \cite{} output to follow that of IEEE. Loading the cite package will
% result in citation numbers being automatically sorted and properly
% "compressed/ranged". e.g., [1], [9], [2], [7], [5], [6] without using
% cite.sty will become [1], [2], [5]--[7], [9] using cite.sty. cite.sty's
% \cite will automatically add leading space, if needed. Use cite.sty's
% noadjust option (cite.sty V3.8 and later) if you want to turn this off.
% cite.sty is already installed on most LaTeX systems. Be sure and use
% version 4.0 (2003-05-27) and later if using hyperref.sty. cite.sty does
% not currently provide for hyperlinked citations.
% The latest version can be obtained at:
% http://www.ctan.org/tex-archive/macros/latex/contrib/cite/
% The documentation is contained in the cite.sty file itself.
%
\usepackage{setspace}
\usepackage{wrapfig}
%\usepackage[usenames, dvipsnames]{color}
%\usepackage[dvipsnames,table,xcdraw]{xcolor}
\usepackage{balance}
\usepackage{mathtools}
\usepackage{multirow}
\usepackage{enumitem}

%\usepackage[pdftex]{graphicx}
\usepackage{graphicx}
% declare the path(s) where your graphic files are
\graphicspath{{./}{./figs}}
% and their extensions so you won't have to specify these with
% every instance of \includegraphics
\DeclareGraphicsExtensions{.pdf,.jpeg,.png}

% *** SUBFIGURE PACKAGES ***
\usepackage[tight,footnotesize]{subfigure}
% \usepackage{subfigure}
% subfigure.sty was written by Steven Douglas Cochran. This package makes it
% easy to put subfigures in your figures. e.g., "Figure 1a and 1b". For IEEE
% work, it is a good idea to load it with the tight package option to reduce
% the amount of white space around the subfigures. subfigure.sty is already
% installed on most LaTeX systems. The latest version and documentation can
% be obtained at:
% http://www.ctan.org/tex-archive/obsolete/macros/latex/contrib/subfigure/
% subfigure.sty has been superceeded by subfig.sty.

%\usepackage[caption=false]{caption}
%\usepackage[font=footnotesize]{subfig}
% subfig.sty, also written by Steven Douglas Cochran, is the modern
% replacement for subfigure.sty. However, subfig.sty requires and
% automatically loads Axel Sommerfeldt's caption.sty which will override
% IEEEtran.cls handling of captions and this will result in nonIEEE style
% figure/table captions. To prevent this problem, be sure and preload
% caption.sty with its "caption=false" package option. This is will preserve
% IEEEtran.cls handing of captions. Version 1.3 (2005/06/28) and later
% (recommended due to many improvements over 1.2) of subfig.sty supports
% the caption=false option directly:
%\usepackage[caption=false,font=footnotesize]{subfig}
%
% The latest version and documentation can be obtained at:
% http://www.ctan.org/tex-archive/macros/latex/contrib/subfig/
% The latest version and documentation of caption.sty can be obtained at:
% http://www.ctan.org/tex-archive/macros/latex/contrib/caption/

% *** PDF, URL AND HYPERLINK PACKAGES ***
%
\usepackage{url}
% url.sty was written by Donald Arseneau. It provides better support for
% handling and breaking URLs. url.sty is already installed on most LaTeX
% systems. The latest version can be obtained at:
% http://www.ctan.org/tex-archive/macros/latex/contrib/misc/
% Read the url.sty source comments for usage information. Basically,
% \url{my_url_here}.

% *** Do not adjust lengths that control margins, column widths, etc. ***
% *** Do not use packages that alter fonts (such as pslatex).         ***
% There should be no need to do such things with IEEEtran.cls V1.6 and later.
% (Unless specifically asked to do so by the journal or conference you plan
% to submit to, of course. )

\usepackage{draftwatermark}

\newcommand{\assign}[1]{\textcolor{red}{(#1)}}
\newcommand{\todo}[1]{\textcolor{orange}{TODO: #1}}
\newcommand{\newtext}[1]{\textcolor{ForestGreen}{#1}}

%Conference
\copyrightyear{2017}
\acmYear{2017}
\setcopyright{usgovmixed}
\acmConference[PDSW-DISCS'17]{PDSW-DISCS'17: Second Joint International Workshop on Parallel Data Storage \& Data Intensive Scalable Computing Systems}{November 12--17, 2017}{Denver, CO, USA}
\acmBooktitle{PDSW-DISCS'17: PDSW-DISCS'17: Second Joint International Workshop on Parallel Data Storage \& Data Intensive Scalable Computing Systems, November 12--17, 2017, Denver, CO, USA}
\acmPrice{15.00}\acmDOI{10.1145/3149393.3149395}
\acmISBN{978-1-4503-5134-8/17/11}

\begin{document}
%\title{Toward Total Knowledge of I/O through Integrated Multicomponent Instrumentation}
\title{UMAMI: A Recipe for Generating Meaningful Metrics through Holistic I/O Performance Analysis}
%GAIL - why use an asterisk for several authors but give only Lockwood's email - shouldn't it be better to use a different footnote symbol for his being corrrepsonding author?
%\author{\IEEEauthorblockN{Glenn K. Lockwood,\IEEEauthorrefmark{1}
%Shane Snyder,\IEEEauthorrefmark{2}
%Wucherl Yoo,\IEEEauthorrefmark{1}
%Kevin Harms,\IEEEauthorrefmark{2} \\
%Zachary Nault,\IEEEauthorrefmark{2}
%Suren Byna,\IEEEauthorrefmark{1}
%Philip Carns,\IEEEauthorrefmark{2}
%Nicholas J. Wright\IEEEauthorrefmark{1}}
%\IEEEauthorblockA{\IEEEauthorrefmark{1}Lawrence Berkeley National Laboratory,
%Berkeley, CA 94720, Email: glock@lbl.gov}
%\IEEEauthorblockA{\IEEEauthorrefmark{2}Argonne National Laboratory, Lemont,
%IL 60439}
%}

\author{Glenn K. Lockwood, Wucherl Yoo, Suren Byna, Nicholas J. Wright}
\email{{glock,wyoo,sbyna,njwright}@lbl.gov}
%\author{Wucherl Yoo}
%\email{glock@lbl.gov}
%\author{Suren Byna}
%\email{glock@lbl.gov}
%\author{Nicholas J. Wright}
%\email{glock@lbl.gov}
\affiliation{Lawrence Berkeley National Laboratory}

\author{Shane Snyder, Kevin Harms, Zachary Nault, Philip Carns}
\email{{ssnyder,carns}@mcs.anl.gov, {harms,znault}@alcf.anl.gov}
%\author{Kevin Harms}
%\email{glock@lbl.gov}
%\author{Zachary Nault}
%\email{glock@lbl.gov}
%\author{Philip Carns}
%\email{glock@lbl.gov}
\affiliation{Argonne National Laboratory}

% shorten author list for headers. 
\renewcommand{\shortauthors}{G. Lockwood et al.}

\renewcommand{\shorttitle}{UMAMI: A Recipe for Generating Meaningful I/O Metrics}

\maketitle

\begin{abstract}

I/O efficiency is essential to productivity in scientific computing, especially as many scientific domains become more data-intensive.
Many characterization tools have been used to elucidate specific aspects of parallel I/O performance, but analyzing individual components of complex I/O subsystems in isolation fails to provide insight into critical questions:
how do the I/O components interact, what are reasonable expectations for application performance, and what are the underlying causes and effects of I/O performance problems?
To address these questions while capitalizing on existing component-level characterization tools,
we propose an approach using on-demand, modular synthesis of \emph{existing} I/O characterization data sources into a unified monitoring and metrics interface (UMAMI) that provides a normalized, coherent, holistic view of I/O behavior. 

In this work, we evaluate the feasibility of this integrative approach by applying it to a month-long benchmarking study conducted on two distinct large-scale computing platforms.
We present three case studies that highlight the importance of analyzing
application I/O performance in context with both contemporaneous and
historical component metrics, and we provide new insights into the factors
affecting I/O performance and how to analyze them.

\end{abstract}


\section{Introduction} \label{sec:introduction}

The stratification of performance and capacity in storage technology is motivating the design of increasingly complex parallel storage systems architectures.
For example, leadership-class computing systems are now being deployed with flash-based, on-fabric burst buffer tiers~\cite{Henseler2016} that provide even higher performance than traditional disk-based scratch file systems~\cite{Bhimji2016}.
Although designed to provide optimal performance and capacity on an economic basis, this increasing number of interoperating components also complicates the task of understanding I/O performance.

The current state of practice is to monitor each component in the I/O stack separately.
However, these components approach I/O from different perspectives, often resulting in component-level monitoring data that are not obviously compatible.
For example, server-side monitoring tools such as LMT~\cite{lmt} measure a limited number of metrics as a high-frequency time series to achieve low overhead, while application-level profiling tools such as Darshan~\cite{carns200924} track metrics that are expressed as bounded
summaries of individual jobs.
% Data types representing the same logical quantity, such as data written, may also be expressed in different units such as bytes, pages, and blocks, and these units of data may also be transformed by aggregation, coalescing, and caching as they traverse the I/O stack.

At present, the gaps of information resulting from these incompatibilities are filled using expert institutional knowledge and intuition.
Absent this expert knowledge though, a principal challenge in understanding I/O performance is knowing how to gauge the performance and behavior of application within a broader context and answer the question: 
does the I/O performance of a given job meet expectations given the capabilities of the system and the nature of the access pattern?
Relying on intuition to answer this question is neither scalable nor sustainable as I/O subsystems of increasing size and complexity are deployed.
Thus, there is a growing need to systematically integrate data from across all components to present a coherent, holistic view of inter-dependent behavior to clarify the relationships that have traditionally fallen on I/O experts.

To demonstrate the new insights possible through an holistic approach, we present a detailed analysis resulting from the combination of data from file system servers, application-level profiling, and other system-level components.
This holistic approach demonstrates that it is possible to differentiate general performance expectations for different I/O motifs (analogous to the climate of the I/O system) from transient effects (analogous to the weather of the I/O system).
We use this notion of the \emph{I/O climate} to encompass the characteristics of storage components, their age and capacity, and the way they generally respond to a specific workload.
Complementary to the I/O climate, the \emph{I/O weather} is determined by transitory state of the job scheduler load, I/O contention, and short-term failure events.

%We show how TOKIO can contextualize I/O performance variation by using it to
%analyze a month-long interval of formalized I/O regression benchmarking on
%two major supercomputing facilities.
%We demonstrate a universal metrics and measurements interface (TOKIO-UMAMI) that quickly classifies a job's I/O performance as being within the expected variation of the file system climate, or if it reflects an extreme file system weather event. 
%Finally, we show how the TOKIO framework can be applied to bridge the gap of expert knowledge by identifying common sources of I/O performance variation.

The primary contributions of this work are as follows:
\begin{itemize}[leftmargin=*]
% \item We describe a framework for holistic instrumentation of I/O subsystems that is generalizable across platforms
\item We integrate the data from existing component-level tools available on NERSC's Edison and ALCF's Mira systems, along with a month-long automated I/O benchmarking effort, to demonstrate novel insights enabled by this holistic approach
\item We show that I/O performance is affected by both intrinsic application characteristics and extrinsic storage system factors.
Contention with other I/O workloads for storage system bandwidth is not the only factor that affects I/O performance, and
we highlight cases where metadata contention and storage capacity both dramatically impacted performance
\item We show that there is no single monitoring metric that predicts I/O performance
universally across HPC platforms; the most highly correlated metrics depend on system architecture, configuration parameters, workload characteristics, and system health.
\end{itemize}


%%%%%%%%%%%%%%%%%%%%%%%%%%%%%%%%%%%%%%%%%%%%%%%%%%%%%%%%%%%%%%%%%%%%%%%%%%%%%%%%
\section{Experimental Methods} \label{sec:platforms}
%%%%%%%%%%%%%%%%%%%%%%%%%%%%%%%%%%%%%%%%%%%%%%%%%%%%%%%%%%%%%%%%%%%%%%%%%%%%%%%%

To examine the utility and generality of integrating multiple component-level monitoring tools' data into a single view (UMAMI), we conducted a month-long experiment where we ran a collection of I/O benchmarks every day on two architecturally distinct computing platforms:
Edison, a Cray XC-30 system at the National Energy Research Scientific Computing Center (NERSC), and Mira, a Blue Gene/Q system at the Argonne Leadership Computing Facility (ALCF).
These benchmarks were run over a period of 29 days (Mira) and 39 days (Edison), during which we collected measurements from the data sources described in Section \ref{sec:methods}.  These data represent a total of 118 (Mira) and 1,014 (Edison) individual benchmark runs.


%and combined these data into UMAMI diagrams to obtain a complete view of I/O during the time each benchmark job ran.
%With this wide body of performance measurements and observed metrics, we characterized the baseline performance variability intrinsic to Edison and Mira (I/O climate), then use this baseline to contextualize the I/O behavior of specific jobs (I/O weather).
%In this section, we detail the configuration of the benchmark suite used and the platforms on which they were run.

% abandon all hope ye who try to edit this stupid table by hand.  I used
% http://www.tablesgenerator.com to make it.
\begin{table}[h]
\footnotesize
\centering
\begin{tabular}{|c|c|c|c|c|c|c|c|}
\hline
 & \textbf{I/O Motif} & \textbf{Mira size} & \textbf{Edison size} \\
\hline
IOR & MPI-IO shared file & 1.0 TiB & 0.5 TiB\\
\hline
IOR & POSIX file per proc & 1.0 TiB & 2.0 TiB\\
\hline
HACC & POSIX file per proc & 1.5 TiB & 2.0 TiB \\
\hline
VPIC / BD-CATS & HDF5 shared file & 1.0 TiB & 2.0 TiB\\
\hline
\end{tabular}
\caption{Benchmark configuration parameters}
\label{tab:bench-config}
\normalsize
\vspace{-.4in}
\end{table}

\subsection{I/O performance regression tests} \label{sec:methods/tests}

%The instrumentation tools described in Section \ref{sec:methods} do not provide a fixed reference point on I/O behavior because
%the workload and state of the system evolve over time.
%To characterize the baseline behavior of Edison and Mira and define the climate of their I/O subsystems, we ran daily I/O performance regression tests based on both synthetic and application-derived workloads.
%We chose 
%a collection of benchmarks because performance is not well-represented
%by a single benchmark result; each storage system has its own strengths
%and weaknesses.  
%The collection of benchmarks is
%executed within a job script that can be scheduled nightly by a continuous
%integration system or cron job.  The script ensures that no more than
%one TOKIO-ABC instance is active at a time, and all benchmark results
%and Darshan logs are archived for analysis at the conclusion of the job.
%The initial set of benchmarks included in TOKIO-ABC are as follows:

We chose to run the following application-derived and synthetic benchmarks at scales sufficient (a) to saturate the storage system while (b) limiting the core-hour consumption to a level tractable for daily execution.
Table \ref{tab:bench-config} summarizes the scale of each application; each benchmark was run using 1,024 compute nodes on Mira and 128 compute nodes on Edison with a fixed 16 processes per node.

The \textbf{Hardware Accelerated Cosmology Code (HACC)} framework~\cite{habib2012} is an N-body cosmology application
for simulating the the evolution of the Universe.
%from its early times to today and for understanding dark energy and dark matter.
We used the HACC-IO kernel which captures HACC's checkpoint I/O and generates 96 MiB/process using POSIX file-per-process I/O.
%Each
%file is written in 10 large chunks, one chunk for each variable.

\textbf{Vector Particle-In-Cell (VPIC)} is a plasma physics application that simulates interactions among billions or trillions of particles \cite{Bowers2008}.
We used the VPIC-IO kernel which captures the I/O operations of a magnetic reconnection simulation, and each MPI process writes $32 \times 2^{20}$ particles (1.0 GiB)
%Each
%particle has eight properties (six floating point and two integer) and each
%property is a single dimension array. 
% The total number of particles depends on
%the number of MPI ranks used. 
to a single shared HDF5 file using the H5Part API \cite{H5Part}.
%and the execution of the I/O kernel includes
%creating and opening a file, writing data to the file, and closing the file.

The \textbf{BD-CATS} clustering system~\cite{Patwary2015} represents
a clustering analysis that is commonly performed on VPIC data.
We configured the BD-CATS I/O kernel to emulate the I/O workload of
a three-dimensional clustering that reads 75\% of the data
contained in the VPIC data file.

\textbf{IOR} is a widely used tool to characterize the performance of parallel file systems\cite{Yildiz2016,Xie2012,Lofstead2010,Uselton2010}.
For the purposes of this work, we applied IOR to determine each file system's performance variability under conditions where an application performs I/O using idealized parameters for each file system.


%%% also used tablesgenerator.com to make this monstrosity
\begin{table*}[h]
\footnotesize
\centering
\begin{tabular}{|c|c|c|c|c|c|c|}
\hline
\textbf{\begin{tabular}[c]{@{}c@{}}System\\ (Center)\end{tabular}}                 & \textbf{Configuration}                                                                                         & \textbf{\begin{tabular}[c]{@{}c@{}}File System\\ (Type)\end{tabular}} & \textbf{\begin{tabular}[c]{@{}c@{}}\# Servers\\ (LUNs)\end{tabular}} & \textbf{\# I/O Nodes}                                                   & \textbf{\begin{tabular}[c]{@{}c@{}}Max\\ Capacity\end{tabular}} & \textbf{\begin{tabular}[c]{@{}c@{}}Peak\\ Performance\end{tabular}} \\ \hline
\multirow{3}{*}{\textbf{\begin{tabular}[c]{@{}c@{}}Edison\\
(NERSC)\end{tabular}}} & \multirow{3}{*}{\begin{tabular}[c]{@{}c@{}}Cray
XC-30\\5,586 CNs\end{tabular}} & scratch1 (Lustre)                                                     & 24 (24)                                                              & 9 (shared)                                                              & 2.2 PB                                                          & 48 GB/sec                                                           \\ \cline{3-7} 
                                                                                   &                                                                                                                & scratch2 (Lustre)                                                     & 24 (24)                                                              & 9 (shared)                                                              & 2.2 PB                                                          & 48 GB/sec                                                           \\ \cline{3-7} 
                                                                                   &                                                                                                                & scratch3 (Lustre)                                                     & 36 (36)                                                              & 13 (shared)                                                             & 3.3 PB                                                          & 72 GB/sec                                                           \\ \hline
\textbf{\begin{tabular}[c]{@{}c@{}}Mira\\ (ALCF)\end{tabular}}                     & \begin{tabular}[c]{@{}c@{}}IBM Blue Gene/Q\\ 49,152 CNs\end{tabular}    & mira-fs1 (GPFS)                                                       & 48 (336)                                                             & \begin{tabular}[c]{@{}c@{}}384 (dedicated)\\ 1 per 128 CNs\end{tabular} & 7.0 PB                                                          & 90 GB/sec                                                           \\ \hline
\end{tabular}
\caption{Description of test platforms}
\label{tab:system-config}
\normalsize
\vspace{-.2in}
\end{table*}

\subsection{NERSC Edison} \label{sec:platforms/edison}

Edison is a Cray XC-30 system deployed at NERSC whose architecture is described in Table \ref{tab:system-config}.
Its scratch1 and scratch2 file systems are identically configured, and users are evenly distributed across both such that the two file systems should have similar levels of baseline I/O traffic.
However, access to Edison's scratch3 file system is only granted to users who require high parallel bandwidth, and therefore the scratch3 file system should reflect larger, more coherent I/O traffic.

The Edison architecture routes I/O traffic from the Cray Aries high-speed network to the FDR InfiniBand-based SAN fabric through LNET I/O nodes.
Routing is configured such that each LNET I/O node handles traffic for only one of the three Edison file systems.
%As a result of this design and the use of Lustre fine-grained routing, all jobs on Edison that utilize one file system share the same set of I/O nodes.
This ensures that each file system's traffic is isolated as it transits I/O nodes and also allows jobs of any size to use the maximum number of I/O nodes for each file system.

For this study, all benchmark data was striped over all of the OSTs in the file system, and the input parameters listed in Table \ref{tab:bench-config} were chosen to saturate each file system's bandwidth.
As such, our IOR benchmarks demonstrated peak performance at 90\% of the theoretical peaks listed in Table \ref{tab:system-config}.

\subsection{ALCF Mira} \label{sec:platforms/mira}

Mira is an IBM Blue Gene/Q system deployed at ALCF whose architecture is detailed in Table \ref{tab:system-config}.
In addition to the servers and LUNs listed, six of the network shared disk (NSD) servers also have an SSD-based LUN on which metadata is stored.
% Mira also has has another primary file system not used in this study, mira-fs0, which shares the same InfiniBand SAN fabric as mira-fs1.
% Although these file systems have independent NSD servers and devices, I/O can still contend for resources on this storage network.

Unlike Edison, Mira's I/O architecture has fixed-size partitions of compute nodes connected to each I/O forwarding node, 
meaning that storage bandwidth available through I/O nodes scales linearly with the size of the compute job.
% Running daily I/O benchmarks that span a large portion of Mira's compute nodes is impractical due to the lengthy queue times of capability jobs and the disruption this would have to job scheduling.
%Therefore the job size used on Mira, 1,024 Mira compute nodes (a single rack), was based upon the availability of compute resources rather than the saturation of I/O subsystem.
To minimize system system disruption and maximize throughput of our daily benchmarks, we opted to run every benchmark using 1,024 compute nodes which 
corresponds to eight I/O nodes and an aggregate peak bandwidth of $\sim$25 GB/sec.
In practice, the IOR configuration for Mira listed in Table \ref{tab:bench-config} was able to achieve 80\% of the peak performance of this I/O node allocation.

%%%%%%%%%%%%%%%%%%%%%%%%%%%%%%%%%%%%%%%%%%%%%%%%%%%%%%%%%%%%%%%%%%%%%%%%%%%%%%%%
\section{Instrumentation \& Data Sources} \label{sec:methods}
%%%%%%%%%%%%%%%%%%%%%%%%%%%%%%%%%%%%%%%%%%%%%%%%%%%%%%%%%%%%%%%%%%%%%%%%%%%%%%%%

\subsection{Application behavior} \label{sec:methods/darshan}

% Application behavior characterization describes the I/O pattern of a workload as expressed from the perspective of the application itself (i.e., before any system-level optimizations are applied).
To capture the I/O patterns and user-observable application performance in this study, we used the Darshan I/O characterization tool~\cite{carns200924} which transparently records statistics about an application's I/O behavior at runtime.
It imposes minimal overhead because it defers the reduction of these statistics until the application exits,
allowing it be deployed for all production applications on large-scale systems without perturbing performance.  Both Mira and Edison link Darshan into all compiled applications by default.
% Because it operates at the application level, Darshan is also highly portable across platforms.

\subsection{Storage system traffic} \label{sec:methods/storagesystraffic}

Storage system traffic monitoring provides insight into the aggregate system-wide I/O workload imposed on a target storage system.
This is reflected in the aggregate bytes read and written, I/O operations processed, and other similar metrics collected at the parallel file system level.
This data can be collected in time-series format with minimal impact to application performance because it is gathered on the storage servers themselves without the involvement of the compute nodes.
On both Mira and Edison, these data were collected at five-second intervals as a balance between low performance impact and sufficient granularity to correlate performance with server activity~\cite{madireddy2017}.  However the two systems employ two different, file system-specific tools.  

\label{sec:methods/lmt}
\textbf{Lustre Monitoring Tool (LMT)} is a tool that aggregates Lustre-specific kernel counters on each Lustre object storage server (OSS) and metadata server (MDS) and presents them to external consumers via a MySQL database.
LMT provides time series data including bytes read and written, CPU load averages, and metadata operation rates.

\label{sec:methods/ggiostat}
\textbf{ggiostat} is a tool developed at ALCF to collect similar data from IBM Spectrum Scale (GPFS) file systems.
It includes a daemon that uses GPFS's \texttt{mmpmon} monitoring system to retrieve and store metrics from server and client clusters, and it provides data including bytes read and written, read and write request counts, and metadata operation counts.

\subsection{Health monitoring} \label{sec:methods/health}

Health monitoring data describe what components are offline, failed-over, or another degraded state, and how much storage space remains on the available devices.
% lfs df
% lctl dl -t
On Edison, the fullness of each Lustre object storage target (OST) is recorded every fifteen minutes.  The server to which each OST is mapped is also logged at this time, allowing us to identify OSTs that have failed over to a partner OSS and are degraded as a result.
% mmlsdisk
% mmdf
On Mira, the fullness of each LUN and the failure status of each server is recorded when each job is submitted.
As with Lustre, recording the mapping between NSD servers and NSDs allows us to identify if a server has failed and a secondary server is handling its NSDs.
% In both GPFS and Lustre cases, we found that these time series data were sufficiently coarse-grained that storing these data in flat files indexed by date was sufficient.

\subsection{Job scheduling} \label{sec:methods/scheduling}

Job scheduling data can provide details on the mix of concurrent jobs that are running on the compute resources of a system to identify cases where I/O contention results from other competing workloads.
% Because job scheduling is most useful in the context of a particular job of interest, we represent this data as a single scalar value that represents the number of other jobs running concurrently with our job of interest.
Because job scheduling is most useful in the context of jobs which overlapped with our benchmark jobs, we counted (2) the number of other jobs that were running, and (2) the number of core hours consumed system-wide during the time each of our benchmark jobs was running.
%we only considered a metric defined as the total number of other jobs running concurrently with our job of interest.

Retrieving these data on Edison is accomplished by querying the job accounting database for all jobs with a start time before the benchmark's end time and an end time after the benchmark's start time.
Similarly, the job accounting data on Mira is stored in a database accessible via a Python API called Ni.
The API allows pulling out a list of jobs for a given time range and running on a specific resource.

\subsection{Job Topology} \label{sec:methods/other}

To identify any effects of job placement on Edison's dragonfly network and Mira's 5D torus, we calculate the maximum radius for a job as a rough approximation of how delocalized the job is on the high-speed network.
By using the topological coordinates of each job's compute node allocation to derive a center of mass of a job, we define this metric as the maximum distance between that center of mass and a compute node.

Job layouts on Edison can be obtained by retrieving job node lists from the Slurm accounting database and joining these job nodes with dragonfly topology coordinates stored the Cray service database.
On Mira, jobs are always close-packed on its torus which results in the maximum radius of each job being identical.

% \section{System-Level Analysis} \label{sec:results}
\section{Baseline I/O Performance} \label{sec:results}

% We begin our analysis with a broad study of the overall I/O climate of each system to
% a) establish baseline expectations for performance and variability and b) formally quantify the relationships between I/O components.
%
%These two factors are critical to interpreting I/O performance, but they have been traditionally documented and communicated in informal anecdotes and institutional expert knowledge.

%%%%%%%%%%%%%%%%%%%%%%%%%%%%%%%%%%%%%%%%%%%%%%%%%%%%%%%%%%%%%%%%%%%%%%%%%%%%%%%%
% \subsection{Baseline I/O Performance} \label{sec:results/overview}
%%%%%%%%%%%%%%%%%%%%%%%%%%%%%%%%%%%%%%%%%%%%%%%%%%%%%%%%%%%%%%%%%%%%%%%%%%%%%%%%

% Figure \ref{fig:perf-summary-boxplots-fs} shows the distribution of performance measured by all benchmarks normalized to the highest performance observed on each file system.
% Mira's distribution of variation is the most narrow, with all benchmarks' 25th percentiles well above 50\% of the peak performance.
% By comparison, shared-file I/O performance (BD-CATS, IOR/shared, and VPIC) on Edison's file systems is appreciably lower than the maximum observed peak, which correspond to file-per-process read workloads (HACC and IOR/fpp).
% Although the source of this disparity in variation between Mira and Edison is not clear from Figure \ref{fig:perf-summary-boxplots-fs} alone, we explore the underlying causes in subsequent sections.

% \begin{figure}[t]
%     \centering
%     \includegraphics[width=1.0\columnwidth]{figs/perf-boxplots-per-fs.pdf}
%     \caption{I/O performance for Edison (scratch1, scratch2, scratch3) and Mira (mira-fs1), normalized to the maximum performance of all tests performed on each file system.
%     Each box reflects the distribution of an application workload and its read or write performance.
%     Whiskers extend to the 5th and 95th percentiles.}
%     \label{fig:perf-summary-boxplots-fs}
% \vspace{-.2in}
% \end{figure}

We first establish the baseline performance variation of each benchmark on each system tested.
% Such variation in peak file system performance caused by different I/O access patterns is well documented~\cite{Lofstead2010,Uselton2010,Xie2012}.
Because variation in peak I/O performance is known to be caused by different I/O access patterns~\cite{Lofstead2010,Uselton2010,Xie2012}, 
% To focus solely on the variation caused by factors \emph{extrinsic} to each application,
we define the \emph{fraction of peak performance} as the performance of a job divided by the maximum performance observed for all jobs \emph{of the same I/O motif} as listed in Table \ref{tab:bench-config} and whether the job did reads or writes.
For example, the fraction peak performance for a HACC write test is only normalized to the maximum performance of all other HACC write tests on the same file system.

The distribution of fraction peak performance, shown in Figure~\ref{fig:perf-summary-boxplots-motif}, reveals that the degree of performance variation \emph{within} each application varies with each file system.
For example, the HACC write workload is susceptible to a long tail of performance degradation on mira-fs1 despite that file system's overall lower variation as evidenced by the distance between all I/O motifs' whiskers relative to the Edison file systems.
% Similarly, all Edison file systems show a long tail of performance loss for the IOR/shared file read workload.
Edison's scratch3 also demonstrates very broad performance variation for the VPIC write workload, contrasting with the relatively narrow performance variation of this application on other systems.

\begin{figure}[t]
    \centering
    \includegraphics[width=1.0\columnwidth]{figs/perf-boxplots.pdf}
    \caption{I/O performance for all file systems tested grouped by test
    applications and read/write mode.  Whiskers represent the 5th and 95th
    percentiles.}
    \label{fig:perf-summary-boxplots-motif}
\vspace{-.2in}
\end{figure}

We can conclude from this that performance variability is the result of factors intrinsic to the application \emph{and} factors intrinsic to the file system;
different I/O motifs result in different levels of performance \emph{and} variability.
Furthermore, these behaviors are not a function of the parallel file system architecture either; all Edison file systems are Lustre-based, yet Figure~\ref{fig:perf-summary-boxplots-motif} shows a marked difference in variability between scratch1/scratch2 and scratch3.
Thus, these differences in performance variation must be a result of their different hardware configurations (discussed in Section \ref{sec:platforms}), their specific user workloads, or a combination of both.
This finding highlights the importance of examining multiple sources of I/O characterization data (e.g., application-level and server-side) in concert to develop a full understanding of I/O performance.

%%%%%%%%%%%%%%%%%%%%%%%%%%%%%%%%%%%%%%%%%%%%%%%%%%%%%%%%%%%%%%%%%%%%%%%%%%%%%%%%
% \subsection{Combined Application/Server Metrics} \label{sec:results/combining}
%%%%%%%%%%%%%%%%%%%%%%%%%%%%%%%%%%%%%%%%%%%%%%%%%%%%%%%%%%%%%%%%%%%%%%%%%%%%%%%%

% \begin{figure}[t]
%     \centering
%     \includegraphics[width=\columnwidth]{figs/cdf-both.pdf}
%     \caption{Cumulative distribution function of the coverage factor (a) and the    performance relative to the maximum throughput observed across each file system (b).
%     The line demarcating 50\% probability corresponds to coverage factors of 0.88, 0.87, 0.84, and 0.94 and peak performance fractions of 0.89, 0.86, 0.78, and 0.95 on Edison scratch1-scratch3 and Mira, respectively.}
%     \label{fig:cdfs}
% \vspace{-.2in}
% \end{figure}

%The distribution of coverage factors across all experiments run are shown in Figure~\ref{fig:cdfs}a which reveals that the majority of tests ($> 75\%$ on Edison and $> 80\%$ on Mira) have high coverage factors ($\mathit{CF} > 0.80$).
%This is consistent with the observation that I/O occurs in bursts~\cite{Carns2011,Liu2016}, and the probability of two bursts coinciding and causing contention for bandwidth (thereby reducing $\mathit{CF}$) is relatively low.
%In particular, Mira's $\mathit{CF}$ distribution is so narrow that over 50\% of tests effectively ran without bandwidth contention; $\mathit{CF} >= 0.99$ corresponds to the 40th percentile on that system.

% Across all benchmark runs, over 75\% (Edison) and 80\% (Mira) of the measurements analyzed had high coverage factors ($\mathit{CF} > 0.80$).
% This is consistent with the observation that I/O occurs in bursts~\cite{Carns2011,Liu2016}, and the probability of two bursts coinciding and causing contention for bandwidth (thereby reducing $\mathit{CF}$) is relatively low.
% In particular, Mira's $\mathit{CF}$ distribution is so narrow that over 50\% of tests effectively ran without bandwidth contention; $\mathit{CF} >= 0.99$ corresponded to the 40th percentile on that system.
% 
% Despite this low incidence of overlapping bursts, the distribution of performance relative to the peak observed bandwidth for each application such as shown in Figure~\ref{fig:perf-summary-boxplots-motif} is more broadly distributed.
% Edison's scratch3 exemplifies this; 26\% of jobs on that file system got less than half of the peak performance (fraction peak performance $< 0.50$) despite only 5\% of jobs showing $\mathit{CF} < 0.50$.  This indicates that the coverage factor (and therefore server-side I/O bandwidth) is not the only contributor to I/O performance degradation.
% This finding is consistent with previous work that found Lustre file system performance to be constrained by both bandwidth \emph{and} I/O operation (IOP) rate~\cite{Uselton2013}.

% \begin{figure}[t]
%     \centering
%     \includegraphics[width=\columnwidth]{figs/hist-cf-bw-and-ops.pdf}
%     \caption{Distribution of the coverage factor for both bandwidth ($\textit{CF}_{\textup{bandwidth}}$) and read/write operations ($\textit{CF}_{\textup{iops}}$) for Mira.
%     }
%     \label{fig:hist-cf-mira}
% \vspace{-.2in}
% \end{figure}
% 
% We can generalize this notion of the coverage factor to any metric for which we can distinguish the contribution of an individual job from the global system-level measurement.
% However, these alternative coverage factor metrics are not likely to correlate well with performance unless the job of interest is making a meaningful contribution to the system-wide load.
% For example, the coverage factor for IOPs can be expressed as the fraction of read and write operations extracted from a job's Darshan log to the total read and write operations logged by ggiostat.  The distribution of this $\textit{CF}_\textit{IOPs}$ metric is shown alongside the bandwidth coverage factor, $\textit{CF}_\textit{BW}$, in  Figure \ref{fig:hist-cf-mira}.
% This figure suggests that the TOKIO-ABC tests likely do not contribute a significant IOPs load Mira;
% the relatively flat distribution of $\textit{CF}_\textit{IOPs}$ is likely a reflection of the background IOPs load which is not significantly perturbed when TOKIO-ABC jobs are running. \todo{is there is a clearer way to state the previous sentence?}
% 

%%%%%%%%%%%%%%%%%%%%%%%%%%%%%%%%%%%%%%%%%%%%%%%%%%%%%%%%%%%%%%%%%%%%%%%%%%%%%%%%
% \subsection{Correlating Performance Metrics} \label{sec:results/correlating}
%%%%%%%%%%%%%%%%%%%%%%%%%%%%%%%%%%%%%%%%%%%%%%%%%%%%%%%%%%%%%%%%%%%%%%%%%%%%%%%%

% Given the observation that bandwidth contention is not the sole source of performance variability, 
% % Although bandwidth is the most intuitive initial metric for application and server correlation,  
% %Our choice to define our correlation parameter according to application performance and server-side bandwidth and IOPs was motivated by a broad body of literature and an intuitive assumption that competition for bandwidth and IOPs affect performance the most dramatically.
% %the TOKIO framework is generalized to draw data from any resource that can be indexed on a per-job or time series basis.
% %Thus,
% % we can 
% we then establish correlations between job I/O performance and other metrics to identify which components in the I/O subsystem are the most likely causes of poor I/O performance.
% To this end, we calculated the Pearson correlation coefficient between each job's fraction of peak performance (defined in \ref{sec:results/overview}) with the range of metrics described in Section \ref{sec:methods}.
% While many of these metrics are not expected to be normally distributed~\cite{Kim2010}, Pearson correlation coefficients are applied here to compare the directions and relative strengths of the relationships between each metric we analyzed.
% As such, this correlation coefficient is a suitable qualitative indicator of general trends and confidences despite the unknown distribution of each underlying component's metrics.
% A summary of some of the most notable correlations (or lack thereof) are presented in Figure \ref{fig:correlation-table} and highlight several interesting findings:
% 
% \begin{figure}[t]
%     \centering
%     \includegraphics[width=\columnwidth]{figs/correlation_table.pdf}
%     \caption{Correlation coefficients between fraction of peak performance measured for each I/O motif and a variety of server-side measurements and metrics.
%     Box color indicates confidence; correlations with p-values $< 0.01$ are blue; p-values $< 0.05$ are green, and p-values $>= 0.05$ are red (and signify lack of correlation).
%     Similarly, \textbf{bolded} values signify moderate correlation ($|r| > 0.30$), and \textit{italicized} values signify weak correlation ($|r| < 0.10$).
%     The "\% Servers Failed Over" metric is not applicable because no
%     changes in server failover state were
%     observed during this study.
%     }
%     \label{fig:correlation-table}
% \vspace{-.2in}
% \end{figure}
% 
% \begin{itemize}[leftmargin=*]
% 
% \item CFs correlate with performance on all file systems, indicating that contending for file system I/O is a moderate contributor to performance loss.
% This correlation is less strong on Mira because the absolute performance of our tests on Mira were limited by the bandwidth of the I/O nodes allocated, leaving bandwidth headroom on the NSD servers for other jobs.
% Mira's file system also demonstrated moderate sensitivity to contention for
% IOPs confirming the generalizability of the conclusions of Uselton and Wright~\cite{Uselton2013} based on Lustre performance.
% %Lustre IOPs were not collected for this study, so no comparison can be made to Edison.
% 
% \item Mira's performance correlates negatively with higher rates of \texttt{open(2)}/\texttt{close(2)} calls than Lustre.
% This is reasonable given that Mira's file system serves metadata from the same physical servers as data.
% By comparison, Edison's file systems each have their own discrete metadata servers that are specifically designed to decouple bulk data transfer performance from metadata operations.
% 
% \item File system fullness has markedly different behavior on Edison relative to Mira.
% While Mira's performance is uncorrelated with device capacity, Edison performance degrades as free space on OSTs is depleted.
% This type of behavior is a known characteristic of the Lustre file system and has been observed on deployments at other HPC centers~\cite{oral2014best}.
% 
% \item Contrary to our expectation, I/O performance shows minimal correlation with the number of other jobs running concurrently.
% The lack of correlation is consistent with our finding that I/O remains highly bursty; a large number of small jobs are highly unlikely to burst simultaneously, and each small job is not individually capable of significantly impacting our tests' coverage factors.
% 
% % \item Similarly, the job diameter (a measurement of how spread out a job is across the compute fabric) has no discernible correlation with I/O performance on Edison.
% % Since the low-diameter dragonfly topology of Edison is designed
% % to reduce the performance impact of job topology, this is consistent with
% % expectation.  Job diameter is not shown for Mira because it utilizes a dense
% % torus partition for each job.
% 
% \end{itemize}
% 
% %We collected additional system-specific measurements that did not correlate significantly to performance.
% %However this correlation analysis was not intended to be exhaustive, and the correlations and p-values for the Edison system are likely diminished by the fact that data for all three Edison file systems were combined for this analysis.

%%%%%%%%%%%%%%%%%%%%%%%%%%%%%%%%%%%%%%%%%%%%%%%%%%%%%%%%%%%%%%%%%%%%%%%%%%%%%%%%
\section{Integrated Analysis} \label{sec:results/umami}
%%%%%%%%%%%%%%%%%%%%%%%%%%%%%%%%%%%%%%%%%%%%%%%%%%%%%%%%%%%%%%%%%%%%%%%%%%%%%%%%

With an understanding of the baseline performance variation on each system and I/O motif tested, we can then combine the application performance data derived from Darshan with server-side load data derived from LMT/ggiostat to
better understand how performance variation is caused by factors extrinsic to the application.
I/O bandwidth contention from other jobs is an intuitive source in I/O performance variation, so we define the \emph{coverage factor} ($\mathit{CF}$) of a job $j$ to quantify the effects of such competing I/O traffic:

\begin{equation} \label{eq:cf}
    \mathit{CF}(j) = \frac{N_{\textup{bytes}}^{\textup{Darshan}}(j)}
    {\sum_{t,s}^{\textup{time,servers}}
    \left [ N_{\textup{bytes}}^{\textup{LMT,ggiostat}}(t,s) \right ] }
\end{equation}
%
where $N_{\textup{bytes}}^{\textup{Darshan}}$ are the bytes read and written by job $j$ from to its Darshan log, and $N_{\textup{bytes}}^{\textup{LMT,ggiostat}}$ are the bytes read and written to a parallel file system server $s$ during a 5-second time interval $t$.
The time interval over which the job ran ($\mathit{time}$) and the servers to which the job wrote ($\mathit{servers}$) are both taken from the job's Darshan log~\cite{snyder2016modular}.
% It should be noted that we can generalize this notion of the coverage factor to any metric for which we can distinguish the contribution of an individual job from the global system-level measurement, but for brevity in this work we focus our analysis on bandwidth coverage factor.

This CF is a direct reflection of how much I/O traffic a job competed against in the underlying file systems.
When $\mathit{CF} = 1.0$, all of the server-side I/O can be attributed to job $j$, while $\mathit{CF} = 0.5$ indicates that only half of the server-side I/O is attributable to job $j$ while the other half is from other sources.
%In practice, $\mathit{CF}$ can be slightly greater than $1.0$ as a result of two conditions:
%a) when the storage system traffic monitoring (LMT/ggiostat) does not capture data from all servers during a polling interval, or
%b) when clock skew between the compute nodes and the file system servers causes the Darshan log and LMT/ggiostat to have an inconsistent understanding of when I/O happened.
%In this study, such noise never resulted in $\mathit{CF} > 1.2$.
%%%% GKL: The ratio of CF > 1.2 to all CF measurements is very high on Mira; 96 of the 214 measurements provided by Shane were dropped due to this filter criterion

This CF metric, along with the system health, job scheduling, and job topology data then allow us to contextualize performance anomalies and describe where on the spectrum of normalcy a job's I/O performance fell relative to jobs with similar motifs, and it can point to the metrics that are most likely culprits for investigation based on historic correlation.

\begin{figure}[t]
    \centering
    \includegraphics[width=1.0\columnwidth]{figs/umami-scratch2-hacc-write.pdf}
    \caption{UMAMI demonstrating the \emph{file system climate} of HACC write workloads on the Edison scratch2 file system compared to a most recent run, which showed highly unusual \emph{file system weather}.
    The left panes show the measurements from previous runs of the same motif, and the box plots in the right panes summarize the distribution of these metrics.
    The star denotes the metrics for the job of interest and is colored according to the quartile in which it falls (red being the worst quartile and blue the best).
    Box plot whiskers extend to the 5th and 95th percentiles, with outliers being denoted as circles.}
    \label{fig:umami-scratch2-hacc-write}
\vspace{-.2in}
\end{figure}

To concisely visualize all of this information, we present a Unified Measurements And Metrics Interface (UMAMI) diagram as a tool to quickly determine how a job of interest's performance compares to similar I/O workloads in the past.
UMAMI, demonstrated in Figure \ref{fig:umami-scratch2-hacc-write}, presents historic measurements (the I/O climate) and summarizes each metric's distribution in an accompanying box plot.
These time series plots terminate with the metrics for the job of interest and define the I/O weather at the time that job ran.
By overlaying this weather on the climate (dashed lines in the box plots), UMAMI provides a quick view of how each metric compared to the statistical distribution of past weather conditions.
With this juxtaposition of a file system's weather and its overall climate, a user can differentiate between a long-term performance problem and a statistically rare event.% analogous to an extreme weather event.

% In the following sections, we illustrate how UMAMI can be applied to the problem of diagnosing poor performance of individual jobs in three case studies.

\subsection{Case Study: I/O Contention}

The specific UMAMI example shown in Figure \ref{fig:umami-scratch2-hacc-write} represents a HACC write test which took place on March 3.
Per the "Job performance" measurement and its value relative to previous instances of this type of job, this job's performance is statistically abnormal.
This poor performance was accompanied by an unusually low coverage factor and high metadata load, and these unfavorable conditions are highlighted as red dashed lines in the box plots that denote their place in the least-favorable quartile of past measurements.
The metrics corresponding to blue dashed lines fell into the most favorable quartile for this problematic job, but they were also not found to correlate with performance.
Thus, we can attribute this HACC job's poor performance to I/O loads extrinsic to this job which competed for both bandwidth and metadata operation rates.

\subsection{Case Study: Metadata Load}

Figure \ref{fig:umami-mira-fs1-vpic-write} represents a VPIC write workload
that showed poor performance on Mira.
Its coverage factor is within normal parameters (orange lines in the box plots denote a value in the second quartile) indicating low bandwidth contention.
Although the IOPS coverage factor is also low, previous conditions have been worse despite a lack of dramatic performance loss (e.g., on March 7).
The only metric that shows a unique, undesirable value is the number of \texttt{readdir(3)} operations handled by the file system.
This is indicative of an expansive file system traversal that was being performed at the same time as the job execution.
% The \emph{readdir(3)} metric was demonstrated to correlate moderately negatively with performance in Figure \ref{fig:correlation-table}.

\begin{figure}[t]
    \centering
    \includegraphics[width=1.0\columnwidth]{figs/umami-mira-fs1-vpic-write.pdf}
    \caption{UMAMI demonstrating the climate surrounding VPIC-IO write workloads on Mira compared to a most recent run, which showed highly unusual weather in the form of an excess of \texttt{readdir(3)} calls.
%   Significance of each pane and its contents are the same as explained in Figure \ref{fig:umami-scratch2-hacc-write}.
    }
    \label{fig:umami-mira-fs1-vpic-write}
\vspace{-.2in}
\end{figure}

%%%%%%%%%%%%%%%%%%%%%%%%%%%%%%%%%%%%%%%%%%%%%%%%%%%%%%%%%%%%%%%%%%%%%%%%%%%%%%%%
\subsection{Case Study: Storage Capacity}
%%%%%%%%%%%%%%%%%%%%%%%%%%%%%%%%%%%%%%%%%%%%%%%%%%%%%%%%%%%%%%%%%%%%%%%%%%%%%%%%

This holistic approach is also able to identify longer-term performance degradation.
Figure \ref{fig:umami-scratch3-hacc-write-long-term} shows the UMAMI view of such an event on Edison's scratch3 file system where coverage factors were not unusual despite an ongoing $2\times$ slowdown over the normal 50 GiB/sec.
The magnitude of performance loss followed the highest CPU load observed across all of the Lustre OSSes almost exactly, and this period coincided also with the scratch3 file system reaching critical levels of fullness.
Although such correlations cannot define causative relationships, these conditions indicated a relationship between critically full storage devices and CPU load (e.g., an increasing cost of scavenging empty blocks) that impacts application performance.
Incidentally, this behavior is consistent with known performance losses that result from Lustre OSTs filling~\cite{oral2014best}.
 
\begin{figure}[t]
    \centering
    \includegraphics[width=1.0\columnwidth]{figs/umami-scratch3-hacc-write-long-term.pdf}
    \caption{UMAMI of HACC write performance on Edison's scratch3 file system showing a longer-term period of performance degradation that was associated with unusually high OSS CPU load.
%   Significance of each pane and its contents are the same as explained in Figure \ref{fig:umami-scratch2-hacc-write}.
    }
    \label{fig:umami-scratch3-hacc-write-long-term}
\vspace{-.2in}
\end{figure}


\section{Related work} \label{sec:related}

Several recent studies have explored how to combine and analyze multiple sources of I/O monitoring information.
Kunkel et al. developed SIOX~\cite{Kunkel:2014:SAC:2769884.2769901} to aggregate information from multiple components of the I/O stack, 
% into a global database with plugins for automatic optimization.
but its reliance on instrumented versions of application libraries to collect metrics makes production deployment challenging.
% Agelastos et al. developed the Lightweight Distributed Metric Service (LDMS)~\cite{7013000} for scalable collection of compute node metrics which could be a data source for similar multicomponent analysis.
%.  The LDMS metrics include client-side I/O counters that could be integrated into TOKIO.
Liu et al. developed AID~\cite{Liu2016} to perform detailed analysis to server-side I/O logs to deduce application-level I/O patterns and make scheduling recommendations.  The clustering approach implemented by AID may be transferable to identifying I/O motifs suitable for generating UMAMI diagrams procedurally.

Other recent studies have explored how to quantify and combat various types of I/O performance variation.
Lofstead et al. observed that variability can be caused by both external interference and internal interference within an application~\cite{Lofstead2010}, and they proposed an adaptive strategy that coordinates I/O activity within an application.
Similarly, Dorier et al. proposed a middleware mechanism for coordinating I/O activity across applications to manage such external interference~\cite{dorier2014calciom}.
Yildiz et al.'s systematic study of I/O interference in a controlled testbed environment found that poor flow control in the I/O path~\cite{Yildiz2016} also contributes to I/O variance, indicating that network-related metrics would be a valuable addition to future work.
Carns et al. reported I/O performance variability for seven frequently repeated production jobs during a two month study~\cite{carns2011understanding} and, similar to our findings, suggested that some access patterns are more susceptible to variability that others.

\section{Conclusions} \label{sec:conclusions}

By integrating data captured with existing tools from applications, storage systems, system health, and job scheduling, we demonstrated that holistically examining all components of the I/O subsystem is essential for understanding I/O performance variation.
We performed a month-long benchmarking study and characterized the I/O \emph{climate} on each system, then presented several case studies to demonstrate instances of abnormal I/O \emph{weather} and their effects on I/O performance.

Integrating metrics into the UMAMI diagram revealed that contention with other I/O workloads for storage system bandwidth, metadata op rates, and storage capacity can, but do not always, impact performance.
No single metric predicts I/O performance across HPC platforms;
the most significant metrics depend on systems' architecture, configuration, workload characteristics, and health, while 
factors such as job radius and $\textit{CF}_{\textit{nodehrs}}$ do not capture enough detail to indicate performance loss.
These findings provide a basis for improving monitoring tools to capture more detailed metrics that can better predict I/O performance.

% The importance of historical performance data in performance analysis also motivates the need for automated classification of jobs based on similar I/O motifs.
% With clustering or similar methods, generating UMAMI diagrams should be easily automated in production environments and made portable across diverse HPC platforms and centers.


\section*{Acknowledgments}
This material is based upon work supported by the U.S. Department of Energy, Office of Science, under contracts DE-AC02-05CH11231 and DE-AC02-06CH11357 (Project: A Framework for Holistic I/O Workload Characterization, Program manager: Dr. Lucy Nowell).
This research used resources of the National Energy Research Scientific Computing Center, a DOE Office of Science
User Facility supported by the Office of Science of the U.S. Department of Energy under Contract No. DE-AC02-05CH11231.
This research also used resources and data generated from resources of the Argonne Leadership Computing Facility, which is a DOE Office of Science User Facility supported under Contract DE-AC02-06CH11357.

\bibliographystyle{ACM-Reference-Format}
%\bibliographystyle{IEEEtran}
\bibliography{REFERENCES}

% \section*{Government License}
% 
% The submitted manuscript has been created by UChicago Argonne, LLC, Operator of Argonne National Laboratory (“Argonne”).
% Argonne, a U.S. Department of Energy Office of Science laboratory, is operated under Contract No. DE-AC02-06CH11357.
% This manuscript has also been authored by an author at Lawrence Berkeley National Laboratory under Contract No. DE-AC02-05CH11231 with the U.S. Department of Energy.
% The U.S. Government retains for itself, and others acting on its behalf, a paid-up nonexclusive, irrevocable worldwide license in said article to reproduce, prepare derivative works, distribute copies to the public, and perform publicly and display publicly, by or on behalf of the Government.
% The Department of Energy will provide public access to these results of federally sponsored research in accordance with the DOE Public Access Plan.
% http://energy.gov/downloads/doe-public-access-plan.

% \appendix

\subsection{Artifact Description} \label{sec:appendix/artifacts}

Consider adding material here per guidance at
\url{http://sc17.supercomputing.org/2017/02/07/submitting-a-technical-paper-to-sc17-participate-in-the-sc17-reproducibility-initiative/}.

\subsubsection{Abstract}

\subsubsection{Description}

\subsubsection{Meta Information}

\paragraph{Obtaining Software}

\paragraph{Hardware Dependencies}

\paragraph{Software Dependencies}

\paragraph{Datasets}

\subsubsection{Installation}

\subsubsection{Experiment Workflow}

\subsubsection{Evaluation and Expected Result}

\subsubsection{Experiment Customization}

\subsubsection{Notes}


\subsection{Computational Results Analysis} \label{sec:appendix/analysis}

Some types of additional diagnostics could be:
\begin{enumerate}
\item validation of the timers (time something with a known execution time, determine the precision and statistical variability of the timer).
\item Confirm results for a manufactured solution.
\item Test the analytics of the problem, e.g., generate a problem with known spectral properties and test its behavior.
\end{enumerate}


\todo{Here's the data we should ultimately provide, though probably not for initial submission:}

\begin{enumerate}
\item software
    \begin{enumerate}
    \item darshan
    \item ggiostat
    \item LMT (do we include if it's not technically ours, or did we sufficiently modify/extend LMT s.t. we can claim it?)
    \item TOKIO-ABC
    \item any of our tools/utilities used for this work?
        \begin{itemize}
        \item e.g., glue/utility scripts for extracting the metrics we are interested in from the data below, etc.
        \item I don't think we could include this stuff in such little time -- these things are cobbled together at the moment and don't embrace the generic, platform-independent narrative we give in the paper
        \item ultimately, i think it would be a good post-paper exercise to go through all of the scripts we've used in the process to think about how to generalize them and refine them so we could package them up in a general tokio repo to be of use to others
        \end{itemize}
    \end{enumerate}

\item data
    \begin{enumerate}
    \item tokio-abc data:
        \begin{itemize}
        \item darshan logs
        \item job logs
        \end{itemize}
    \item lmt/ggiostats data for the date range of our study
    \item other data we present: health monitoring, scheduler data, etc.
    \end{enumerate}
\end{enumerate}


\end{document}
