\documentclass[conference,10pt,compsocconf]{IEEEtran}

% *** CITATION PACKAGES ***
%
\usepackage{cite}
% cite.sty was written by Donald Arseneau
% V1.6 and later of IEEEtran pre-defines the format of the cite.sty package
% \cite{} output to follow that of IEEE. Loading the cite package will
% result in citation numbers being automatically sorted and properly
% "compressed/ranged". e.g., [1], [9], [2], [7], [5], [6] without using
% cite.sty will become [1], [2], [5]--[7], [9] using cite.sty. cite.sty's
% \cite will automatically add leading space, if needed. Use cite.sty's
% noadjust option (cite.sty V3.8 and later) if you want to turn this off.
% cite.sty is already installed on most LaTeX systems. Be sure and use
% version 4.0 (2003-05-27) and later if using hyperref.sty. cite.sty does
% not currently provide for hyperlinked citations.
% The latest version can be obtained at:
% http://www.ctan.org/tex-archive/macros/latex/contrib/cite/
% The documentation is contained in the cite.sty file itself.
%
\usepackage{setspace}
\usepackage{wrapfig}
\usepackage{color}
\usepackage{balance}

%\usepackage[pdftex]{graphicx}
\usepackage{graphicx}
% declare the path(s) where your graphic files are
\graphicspath{{./}{./figs}}
% and their extensions so you won't have to specify these with
% every instance of \includegraphics
\DeclareGraphicsExtensions{.pdf,.jpeg,.png}

% *** SUBFIGURE PACKAGES ***
\usepackage[tight,footnotesize]{subfigure}
% \usepackage{subfigure}
% subfigure.sty was written by Steven Douglas Cochran. This package makes it
% easy to put subfigures in your figures. e.g., "Figure 1a and 1b". For IEEE
% work, it is a good idea to load it with the tight package option to reduce
% the amount of white space around the subfigures. subfigure.sty is already
% installed on most LaTeX systems. The latest version and documentation can
% be obtained at:
% http://www.ctan.org/tex-archive/obsolete/macros/latex/contrib/subfigure/
% subfigure.sty has been superceeded by subfig.sty.

%\usepackage[caption=false]{caption}
%\usepackage[font=footnotesize]{subfig}
% subfig.sty, also written by Steven Douglas Cochran, is the modern
% replacement for subfigure.sty. However, subfig.sty requires and
% automatically loads Axel Sommerfeldt's caption.sty which will override
% IEEEtran.cls handling of captions and this will result in nonIEEE style
% figure/table captions. To prevent this problem, be sure and preload
% caption.sty with its "caption=false" package option. This is will preserve
% IEEEtran.cls handing of captions. Version 1.3 (2005/06/28) and later
% (recommended due to many improvements over 1.2) of subfig.sty supports
% the caption=false option directly:
%\usepackage[caption=false,font=footnotesize]{subfig}
%
% The latest version and documentation can be obtained at:
% http://www.ctan.org/tex-archive/macros/latex/contrib/subfig/
% The latest version and documentation of caption.sty can be obtained at:
% http://www.ctan.org/tex-archive/macros/latex/contrib/caption/

% *** PDF, URL AND HYPERLINK PACKAGES ***
%
\usepackage{url}
% url.sty was written by Donald Arseneau. It provides better support for
% handling and breaking URLs. url.sty is already installed on most LaTeX
% systems. The latest version can be obtained at:
% http://www.ctan.org/tex-archive/macros/latex/contrib/misc/
% Read the url.sty source comments for usage information. Basically,
% \url{my_url_here}.

% *** Do not adjust lengths that control margins, column widths, etc. ***
% *** Do not use packages that alter fonts (such as pslatex).         ***
% There should be no need to do such things with IEEEtran.cls V1.6 and later.
% (Unless specifically asked to do so by the journal or conference you plan
% to submit to, of course. )

\begin{document}
\title{A Holistic Approach to Characterizing HPC I/O Workloads}

\maketitle

\begin{abstract}

PHC: at the risk of speculating too much about results before we have them:

A variety of instrumentation and analysis tools have been utilized to
great effect to help understand and optimize components such as disk
arrays, file servers, libraries, and applications.  However, these
tools operate in isolation with a limited view of the system as a whole.
An integrated, holistic approach to I/O characterization is needed on
today's increasingly complex systems in order to fully understand
the interactions between components.  These complex interactions are what 
ultimately dictate overall scientific productivity in practice.

In this work we explore a case study in holistic I/O characterization
using an exemplar scientific computing application [description(s?)] on
Cori [description].  Our case study brings together characterization from
X,Y,Z to illustrate how [application performance?  system performance?]
could be improved via integrated holistic I/O characterization.
We make THING A go B \% faster for a meaningful workload.  We evaluate
our instrumentation methods and find that this approach is not only
highly valuable for case studies, but is also entirely feasible for
ongoing 24/7 production deployment without perturbing applications [some
empirical backup].  Finally, we propose a set of recommendations for
how this methodology can be automated in the future.

\end{abstract}

\section{Introduction}

I/O is hard and burst buffers are hard.

\section{NOTES}

An attempt at transcribing notes from our ongoing email discussions:

\section{Background}

Darshan is a lightweight I/O characterization tool~\cite{carns200924}.

LMT.

Datawarp.

\section{Methods}

We ran an example benchmark on Cori with and without burst buffer.

\subsection{I/O application benchmark (Suren)}

We ran Suren's application I/O benchmark, VPIC-IO.

\subsection{System description (Glenn)}

We used Cori, NERSC's Cray XC40 featuring 144 burst buffer nodes sprayed
across the dragonfly network.

\subsection{Data sources (Glenn)}

We collected data for the job time intervals from any sources we could think
of, including:

\begin{itemize}
\item Darshan
\item LMT
\item SLURM job log
\item hopefully some DVS metrics
\item ???
\end{itemize}

\section{Results}

We discovered many wonderful things.

\subsection{Application performance improvement on burst buffer}

This is where we put some pretty charts showing the performance comparison.
Also Phil's fourth bullet point:

\begin{itemize}
\item learn something interesting about the burst buffer and how it integrates
into the ecosystem
\end{itemize}

\subsection{Challenges in integrating data}

This will be more a methods section that calls out Phil's first two bullet points:

\begin{itemize}
\item identify gaps or things that need to be changed in one or more of the tools
to help make sense of the data
\item start getting ideas for how we would integrate the data in an automated
sense

We can also discuss the practicalities of extracting the leading factors
reflective of I/O performance (Jialin's idea).
\end{itemize}


\section{Conclusions}

We know a little more about I/O.

\bibliographystyle{IEEEtran}
\bibliography{REFERENCES}

\end{document}
